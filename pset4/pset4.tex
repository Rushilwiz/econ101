\documentclass{article}

\usepackage{fancyhdr}
\usepackage{extramarks}
\usepackage{amsmath}
\usepackage{amsthm}
\usepackage{amsfonts}
\usepackage{tikz}
\usepackage[plain]{algorithm}
\usepackage{algpseudocode}
\usepackage[shortlabels]{enumitem}
\usepackage{mathtools}
\usepackage{amssymb}
\usepackage{hyperref}
\usepackage{tikz}
\usepackage{pgfplots}

\usetikzlibrary{automata,positioning}

%
% Basic Document Settings
%

\topmargin=-0.45in
\evensidemargin=0in
\oddsidemargin=0in
\textwidth=6.5in
\textheight=9.0in
\headsep=0.25in

\linespread{1.1}

\pagestyle{fancy}
\lhead{\hmwkAuthorName}
\chead{\hmwkClassTime (\hmwkClassInstructor): \hmwkTitle}
\lfoot{\lastxmark}
\cfoot{\thepage}

\renewcommand\headrulewidth{0.4pt}
\renewcommand\footrulewidth{0.4pt}

\setlength\parindent{0pt}

%
% Create Problem Sections
%

\newcommand{\enterProblemHeader}[1]{
    \nobreak\extramarks{}{Problem \arabic{#1} continued on next page\ldots}\nobreak{}
    \nobreak\extramarks{Problem \arabic{#1} (continued)}{Problem \arabic{#1} continued on next page\ldots}\nobreak{}
}

\newcommand{\exitProblemHeader}[1]{
    \nobreak\extramarks{Problem \arabic{#1} (continued)}{Problem \arabic{#1} continued on next page\ldots}\nobreak{}
    \stepcounter{#1}
    \nobreak\extramarks{Problem \arabic{#1}}{}\nobreak{}
}

\setcounter{secnumdepth}{0}
\newcounter{partCounter}
\newcounter{homeworkProblemCounter}
\setcounter{homeworkProblemCounter}{1}
\nobreak\extramarks{Problem \arabic{homeworkProblemCounter}}{}\nobreak{}

\newcommand{\hmwkTitle}{Problem Set 4}
\newcommand{\hmwkDueDate}{April 12, 2024}
\newcommand{\hmwkClass}{Introduction to Economics}
\newcommand{\hmwkClassTime}{ECON 101}
\newcommand{\hmwkClassInstructor}{Robert McDonough}
\newcommand{\hmwkAuthorName}{\textbf{Rushil Umaretiya}}

%
% Title Page
%

\title{
    \vspace{2in}
    \textmd{\textbf{\hmwkClass:\ \hmwkTitle}}\\
    \normalsize\vspace{0.1in}\small{\textbf{Due\ on\ \hmwkDueDate\ at 11:59pm}}\\
    \normalsize\text{Tuesday/Thursday 3:30-4:45, Genome Sciences 100}\\
    \vspace{0.1in}\large{\textit{\hmwkClassInstructor\ - \hmwkClassTime}}
    \vspace{3in}
}

\author{\hmwkAuthorName\\\small{rumareti@unc.edu}}
\date{}

\renewcommand{\part}[1]{\textbf{\large Part \Alph{partCounter}}\stepcounter{partCounter}\\}

%
% Various Helper Commands
%

% Useful for algorithms
\newcommand{\alg}[1]{\textsc{\bfseries \footnotesize #1}}

% For derivatives
\newcommand{\deriv}[1]{\frac{\mathrm{d}}{\mathrm{d}x} (#1)}

% For partial derivatives
\newcommand{\pderiv}[2]{\frac{\partial}{\partial #1} (#2)}

% Integral dx
\newcommand{\dx}{\mathrm{d}x}

% Alias for the Solution section header
\newcommand{\solution}{\textbf{\large Solution}}

\newcommand{\question}[1]{\pagebreak\section{Question #1}}

% Probability commands: Expectation, Variance, Covariance, Bias
\newcommand{\E}{\mathrm{E}}
\newcommand{\Var}{\mathrm{Var}}
\newcommand{\Cov}{\mathrm{Cov}}
\newcommand{\Bias}{\mathrm{Bias}}

\begin{document}

\maketitle

\question{1}

Use the AD-AS model to forecast how the economy will respond to
each of the following scenarios. For each scenario (1) indicate whether
AD or AS will shift, and in which direction; (2) draw a graph illustrating how the economy will respond in the short-run; (3) characterize
what the short-run equilibrium means for the economy (e.g., will it
create inflation, a recession, etc.)

\begin{enumerate}[(a)]
    \item Due to rapid economic development in India, Ford pickup trucks produced in America become very popular and exports rise.
    \item A computer virus is released that targets automated industrial machinery, severely limiting the amount of production by American manufacturing firms for about 2 months.
    \item In 2017 President Trump signed a tax bill that reduced income taxes for many Americans. Now, in 2023, some of those tax cuts are expiring, meaning that income taxes for Americans may go up in 2024.
    \item With the new GDP report, American businesses become more confident that a recession is not approaching, and the economic expansion is likely to continue for the foreseeable future.
\end{enumerate}

\pagebreak
\question{2}
In this question, you will explore how changes in the economy develop
over time. Assume we start in LR equilibrium. Suppose we see the
consumer confidence index rise. Use our AD-AS model to discuss the
following:

\begin{enumerate}[(a)]
    \item Would this event impact AD, SRAS, or LRAS? Show the impact of this in the short-run on an AD-AS graph.
    \item Would this event create an output gap? If so, what kind? Indicate this on your graph.
    \item Has this event changed long-run potential output? If so, has it shifted up, or down?
    \item How would the market move from short-run to long-run equilibrium? Visualize this on your graph.
    \item How would the market move from short-run to long-run equilibrium? Visualize this on your graph.
\end{enumerate}

\pagebreak

\question{3}
During the Coronavirus pandemic, U.S. policymakers used several different levers to try to limit the damage to the economy. For each of
the following policy changes, (1) decide if this was fiscal or monetary
policy, (2), specify how this policy change is designed to impact the
economy, and (3) answer the follow-up question given for each.

\begin{enumerate}[(a)]
    \item Between March 3, 2021 and March 15, 2021, the Federal Reserve announced that it would alter the discount rate (and other interest rates that it controls) in order to push the federal funds rate to 0\%.
    \item The Federal Reserve also announced on March 15th that it would keep the federal funds rate near 0\% "until it is confident that the economy has weathered recent events and is on track to achieve its maximum employment and price stability goals."
    \item The U.S. Congress sent taxpayers stimulus checks worth \$1,800 in 2020, and then \$1,200 in 2021.
    \item The federal government expanded unemployment insurance in the U.S., increasing the length of time that individuals could receive unemployment insurance and also increasing the amount of money that people could receive via unemployment insurance.
\end{enumerate}

\pagebreak

\question{4}

Robert is thinking about taking a vacation to Europe this summer, to
visit old friends in Germany and Austria. Since these countries use
the Euro for currency, Robert will need to purchase some Euros (€)
with dollars.

\begin{enumerate}[(a)]
    \item What is the exchange rate between dollars and Euros as of October 27th, 2023?
    \item If Robert wants to purchase 1,000 €, how much will it cost him in dollars?
    \item Now consider the daily market for Euros. Given the information you found in (a), graph the market for Euros. Be sure to specify the price in this market. Assume that around 1 billion Euros are purchased each day using dollars.
    \item Imagine that many, many Americans have the same idea as Robert to vacation in Europe this summer, so that the demand for Euros increases. Graph how this change will impact the market for Euros. Be sure to indicate how the price and quantity of Euros being traded for dollars is changing.
    \item Explain how the change in the price of a Euro will lead the quantity of Euros supplied to change.
\end{enumerate}

\pagebreak

\question{5}

The U.S. produces some computer graphics chips domestically, but
imports most of them from abroad. Assume that the world market for
graphics chips is competitive and that the U.S. is a small producer,
unable to affect the world price. The world price for computer graphics
chips is \$500. Since the U.S. imports graphics chips, we know that
in the absence of trade, the U.S. equilibrium price on graphics chips
would exceed the world price. Specifically, in the absence of trade, the
equilibrium price for a graphics chip in the U.S. would be \$600.

\begin{enumerate}[(a)]
    \item Draw a diagram showing the U.S. market for graphics chips, given the world market for graphics chips that the U.S. can access.
    \item On your graph above, indicate where we can see the U.S. production of graphics chips, and where we can see the quantity of graphics chips being imported.
    \item Assuming they are identical in terms of capabilities, will the domestically produced graphics chips sell for the same price as the imported ones? Why or why not?
    \item Imagine that the U.S. imposes a \$50 tariff on graphics chips. Graph the impact of this tariff on the U.S. market. Indicate where we can see the domestic production of graphics chips, as well as any imported graphics chips (if any are still imported).
    \item Including the cost of the tariff, how will the cost of a U.S. graphics chip compare to the cost of an imported graphics chip? Explain your answer.
    \item Who benefits and who is harmed by such a tariff? Show these effects in your diagram.
\end{enumerate}

\end{document}