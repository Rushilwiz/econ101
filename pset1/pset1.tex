\documentclass{article}

\usepackage{fancyhdr}
\usepackage{extramarks}
\usepackage{amsmath}
\usepackage{amsthm}
\usepackage{amsfonts}
\usepackage{tikz}
\usepackage[plain]{algorithm}
\usepackage{algpseudocode}
\usepackage[shortlabels]{enumitem}
\usepackage{mathtools}
\usepackage{amssymb}

\usetikzlibrary{automata,positioning}

%
% Basic Document Settings
%

\topmargin=-0.45in
\evensidemargin=0in
\oddsidemargin=0in
\textwidth=6.5in
\textheight=9.0in
\headsep=0.25in

\linespread{1.1}

\pagestyle{fancy}
\lhead{\hmwkAuthorName}
\chead{\hmwkClassTime (\hmwkClassInstructor): \hmwkTitle}
\lfoot{\lastxmark}
\cfoot{\thepage}

\renewcommand\headrulewidth{0.4pt}
\renewcommand\footrulewidth{0.4pt}

\setlength\parindent{0pt}

%
% Create Problem Sections
%

\newcommand{\enterProblemHeader}[1]{
    \nobreak\extramarks{}{Problem \arabic{#1} continued on next page\ldots}\nobreak{}
    \nobreak\extramarks{Problem \arabic{#1} (continued)}{Problem \arabic{#1} continued on next page\ldots}\nobreak{}
}

\newcommand{\exitProblemHeader}[1]{
    \nobreak\extramarks{Problem \arabic{#1} (continued)}{Problem \arabic{#1} continued on next page\ldots}\nobreak{}
    \stepcounter{#1}
    \nobreak\extramarks{Problem \arabic{#1}}{}\nobreak{}
}

\setcounter{secnumdepth}{0}
\newcounter{partCounter}
\newcounter{homeworkProblemCounter}
\setcounter{homeworkProblemCounter}{1}
\nobreak\extramarks{Problem \arabic{homeworkProblemCounter}}{}\nobreak{}

\newcommand{\hmwkTitle}{Problem Set 1}
\newcommand{\hmwkDueDate}{January 26th, 2024}
\newcommand{\hmwkClass}{Introduction to Economics}
\newcommand{\hmwkClassTime}{ECON 101}
\newcommand{\hmwkClassInstructor}{Robert McDonough}
\newcommand{\hmwkAuthorName}{\textbf{Rushil Umaretiya}}

%
% Title Page
%

\title{
    \vspace{2in}
    \textmd{\textbf{\hmwkClass:\ \hmwkTitle}}\\
    \normalsize\vspace{0.1in}\small{\textbf{Due\ on\ \hmwkDueDate\ at 11:59pm}}\\
    \normalsize\text{Tuesday/Thursday 3:30-4:45, Genome Sciences 100}\\
    \vspace{0.1in}\large{\textit{\hmwkClassInstructor\ - \hmwkClassTime}}
    \vspace{3in}
}

\author{\hmwkAuthorName\\\small{rumareti@unc.edu}}
\date{}

\renewcommand{\part}[1]{\textbf{\large Part \Alph{partCounter}}\stepcounter{partCounter}\\}

%
% Various Helper Commands
%

% Useful for algorithms
\newcommand{\alg}[1]{\textsc{\bfseries \footnotesize #1}}

% For derivatives
\newcommand{\deriv}[1]{\frac{\mathrm{d}}{\mathrm{d}x} (#1)}

% For partial derivatives
\newcommand{\pderiv}[2]{\frac{\partial}{\partial #1} (#2)}

% Integral dx
\newcommand{\dx}{\mathrm{d}x}

% Alias for the Solution section header
\newcommand{\solution}{\textbf{\large Solution}}

\newcommand{\question}[1]{\pagebreak\section{Question #1}}

% Probability commands: Expectation, Variance, Covariance, Bias
\newcommand{\E}{\mathrm{E}}
\newcommand{\Var}{\mathrm{Var}}
\newcommand{\Cov}{\mathrm{Cov}}
\newcommand{\Bias}{\mathrm{Bias}}

\begin{document}

\maketitle

\question{1}

We apply the cost-benefit principle every day. As students, the choice to attend UNC involved big costs and (hopefully) benefits.

\begin{enumerate}[(a)]

    \item Attending UNC involves large out-of-pocket costs, which are listed
    on UNC's student aid website:
    https://studentaid.unc.edu/current/costs/.
    Using this site, find the total yearly cost of attending UNC as an
    in-state student.

    \item What is the North Carolina state minimum wage? Use the state minimum wage to calculate the foregone wages that you lose by attending UNC for the year.

    \item Not all costs are measured in dollars! Describe some of the nonmonetary costs of spending a year at UNC.

    \item At UNC, most students graduate after 8 semesters (4 years). Setting aside the non-monetary costs, use the numbers you found
    above to calculate the opportunity cost of earning your degree. Ignore the possibility of student loans and aid, and pretend that
    you are paying out of pocket.

    \item Explain the cost-benefit principle in a sentence or two. Incorporating the numbers you found above, then explain your decision
    to attend UNC this year using the cost-benefit principle.
\end{enumerate}

\question{2} Your car needs gas before you can go to work this morning. You
decide to go to the gas station that is out of the way, but where gas
is \$0.10/gallon cheaper than the gas station on the way to work. This
gets you into work 10 minutes later than going to the other gas station.
If your wage is \$20/hour and you have to purchase 20 gallons of gas,
was this worth it? Why or why not?

\question{3} Tanner and Jasmine are each capable of producing two services: walking dogs or cooking meals. Tanner can cook a meal for 6 people in
an hour, or walk 1 dog in an hour. Jasmine can cook a meal for 2
person in an hour, or walk 3 dogs in an hour. They each have 4 hours
available to use to cook meals or walk dogs.
\begin{enumerate}[(a)]
    \item Draw a production possibilities frontier showing Tanner's capacity to cook meals or walk dogs, then add another PPF showing
Jasmine's ability to cook meals or walk dogs.

    \item Label (including numbers) a point on Tanner's PPF that he could
produce at without trading. Do the same for a point on Jasmine's
PPF

    \item Who has the comparative advantage in cooking meals? Who
has the comparative advantage in walking dogs? Explain both
answers.

    \item Tanner and Jasmine decide to specialize in producing one thing,
then trade. What will Tanner choose to produce and what will
Jasmine choose to produce. Explain your answer.

    \item What can we say about the price that Tanner and Jasmine would
both be willing pay to trade meals and dog walks?

    \item Suppose that before trading, Tanner and Jasmine each spent two
hours walking dogs and two hours cooking meals. What are the
gains to specialization and trade in this situation? Provide an
example for how the gains from trade could be distributed so
that Tanner and Jasmine each have more of each service than
before.

    \item In your example for how the gains of trade could be distributed,
how much of each good are Tanner and Jasmine trading to one
another? Do these ”terms of trade” make sense, given what you
wrote in part (e)?
\end{enumerate}

\question{4}

Consider the market for a new physical copy of our textbook, \emph{Principles of Economics by Stevenson and Wolfers}. The instructors teaching large classes of ECON 101 at UNC all use this textbook. For each
of the following situations, decide if demand will shift, if supply will
shift, or if neither will shift. Then, draw a graph clearly illustrating
how supply or demand will shift.
\begin{enumerate}[(a)]
    \item The price of textbook ink increases.

    \item UNC mandates that all arts and science majors must take ECON
101.

    \item The price of the textbook rises.

    \item The price of used copies of the old edition of the textbook decrease.
\end{enumerate}

\question{5}

Consider the daily market for a cup of coffee in Chapel Hill. Market
demand for coffee is given by the equation \(P = 80 - \frac{1}{2}Q_d\), and market
supply of coffee is given by \(P = \frac{Q_s}{38}\).
\begin{enumerate}[(a)]
    \item If the price of coffee is \$0, how many cups would buyers want to
consume? How many cups would sellers want to sell?

    \item Calculate the price at which buyers would not want to buy any
coffee (i.e., \(Q_d = 0\)).

    \item Calculate the equilibrium price of coffee and the quantity of coffee
cups sold in Chapel Hill every day.

    \item Draw a properly labeled diagram for the market for coffee in
Chapel Hill.

\end{enumerate}

\end{document}